\documentclass[11pt]{article}
\usepackage{amsmath, amssymb, fancyhdr, fullpage, listings, enumitem}
\usepackage[usenames,dvipsnames]{xcolor}
\usepackage{fancyvrb}
\usepackage{titlesec}
\titleformat{\section}{\normalfont\Large\bfseries}{}{0em}{}

\setlength{\parindent}{0pt}
\setlength{\parskip}{0.75em}

\title{CPSC 250 - Programming for Data Manipulation\\Test 3}
\date{}
\begin{document}
\maketitle

\textbf{Instructions:} Answer all questions on this test paper. You may not use any devices. Write clearly and show your work where needed.

\section*{Part 1 – Multiple Choice (10 points)}

Circle the best answer. Each question is worth 2 points.

\begin{enumerate}[label=\arabic*.]
    \item What will happen if a derived class does \textbf{not} call \texttt{super().\_\_init\_\_()} in its constructor?
    \begin{enumerate}[label=\alph*)]
        \item The base class is automatically initialized
        \item The base class’s constructor is ignored
        \item Python creates a default constructor for it
        \item It has no effect on object creation
    \end{enumerate}

    \item Which of the following is \textbf{true} about overriding methods in a derived class?
    \begin{enumerate}[label=\alph*)]
        \item You must use the same method name but different parameters
        \item You must redefine all parent methods
        \item The method in the base class is permanently deleted
        \item The derived method replaces the base version when called from an instance
    \end{enumerate}

    \item In object-oriented programming, what is \textbf{polymorphism} most commonly used for?
    \begin{enumerate}[label=\alph*)]
        \item To allow different object types to respond to the same method call
        \item To copy attributes from parent to child classes
        \item To convert private attributes to public
        \item To compare two unrelated types for equality
    \end{enumerate}
    
    \newpage

    \item Consider the following:

\begin{verbatim}
class A:
    def speak(self):
        return "A speaks"

class B(A):
    def speak(self):
        return "B speaks"

class C(A):
    def speak(self):
        return "C speaks"

class D(B, C):
    pass

print(D().speak())
\end{verbatim}

What is printed?
    \begin{enumerate}[label=\alph*)]
        \item A speaks
        \item B speaks
        \item C speaks
        \item Error due to ambiguity
    \end{enumerate}

    \item What is the main purpose of using \textbf{encapsulation} in object-oriented design?
    \begin{enumerate}[label=\alph*)]
        \item To allow multiple classes to inherit the same method
        \item To ensure subclasses always override base methods
        \item To restrict direct access to some parts of an object
        \item To define attributes in the constructor
    \end{enumerate}
\end{enumerate}

\newpage

\section*{Part 2 – Find the Errors (10 points)}

Each question contains at least two errors related to inheritance or object design. Circle the errors and explain briefly what is wrong.

\subsection*{Question 1:}
\begin{verbatim}
class Vehicle:
    def __init__(self, speed):
        speed = speed

class Bike(Vehicle):
    def __init__(self, speed, gear):
        self.gear = gear
\end{verbatim}

\vspace{3cm}

\subsection*{Question 2:}
\begin{verbatim}
class Tool:
    def __init__(self, name):
        self.name = name

class Hammer(Tool):
    def __init__(self, name, weight):
        Tool.__init__()
        self.weight = weight
\end{verbatim}

\vspace{3cm}

\newpage

\section*{Part 3 – Code Writing (30 points)}

Answer each of the following questions clearly. Be sure to use correct syntax and indentation.

\subsection*{1. (10 points) – Inheritance with \texttt{super()} and \texttt{\_\_str\_\_}}

Write a class \texttt{Book} with attributes \texttt{title} and \texttt{author}. Then write a subclass \texttt{Textbook} that adds a \texttt{subject} attribute. Use \texttt{super()} to initialize the base class, and override \texttt{\_\_str\_\_} to return:

\texttt{"Calculus by Stewart [Subject: Math]"}

\vspace{6cm}
\newpage

\subsection*{2. (10 points) – \texttt{\_\_eq\_\_} and \texttt{\_\_lt\_\_}}

Create a class \texttt{Movie} with attributes \texttt{title} and \texttt{rating} (a float from 0 to 10).

\begin{itemize}
    \item Override \texttt{\_\_eq\_\_} so two movies are equal if their titles match
    \item Override \texttt{\_\_lt\_\_} so movies can be compared by rating
\end{itemize}

\vspace{6cm}
\newpage

\subsection*{3. (10 points) – Mixin + Inheritance}

Write a mixin class \texttt{TimestampMixin} that defines a method \texttt{timestamp()} which prints "Logged at some time". Create a base class \texttt{Document} with an attribute \texttt{filename}. Write a class \texttt{PDFDocument} that inherits from both and includes a \texttt{print\_info()} method that prints the filename and calls \texttt{timestamp()}.

\vspace{6cm}

\newpage

\section*{Part 4 – Comment the Code (10 points)}

Write meaningful comments on the lines marked below. Don’t just rewrite the code — explain what each part is doing and why.

\begin{verbatim}
class Animal:
    def __init__(self, species="unknown"):       # (1)
        self._species = species

    def get_species(self):                        # (2)
        return self._species

    def speak(self):                              # (3)
        return "..."

class Dog(Animal):
    def speak(self):                              # (4)
        return "Woof!"
\end{verbatim}

\vspace{5cm}

\end{document}
