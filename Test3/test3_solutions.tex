\documentclass[12pt]{article}
\usepackage{amsmath, amssymb, fullpage, fancyhdr, listings, enumitem}
\usepackage[usenames,dvipsnames]{xcolor}
\usepackage{fancyvrb}
\usepackage{titlesec}
\titleformat{\section}{\normalfont\Large\bfseries}{}{0em}{}

\pagestyle{fancy}
\fancyhf{}
\rhead{CPSC 250 – Test 3 Solutions}
\lhead{}
\rfoot{Page \thepage}

\setlength{\parindent}{0pt}
\setlength{\parskip}{0.75em}

\title{CPSC 250 - Programming for Data Manipulation\\Test 3 -- \textbf{Solution Set}}
\date{}
\begin{document}
\maketitle

\section*{Part 1 – Multiple Choice (10 points)}

\begin{enumerate}[label=\arabic*.]
    \item \textbf{b)} The base class’s constructor is ignored
    \item \textbf{d)} The derived method replaces the base version when called from an instance
    \item \textbf{a)} To allow different object types to respond to the same method call
    \item \textbf{b)} B speaks
    \item \textbf{c)} To restrict direct access to some parts of an object
\end{enumerate}

\section*{Part 2 – Find the Errors (10 points)}

\subsection*{Question 1:}
\begin{verbatim}
class Vehicle:
    def __init__(self, speed):
        speed = speed   # Error: should be self.speed = speed

class Bike(Vehicle):
    def __init__(self, speed, gear):
        self.gear = gear  # Error: base class constructor not called
\end{verbatim}

\textbf{Fix:}
\begin{verbatim}
class Bike(Vehicle):
    def __init__(self, speed, gear):
        super().__init__(speed)
        self.gear = gear
\end{verbatim}

\subsection*{Question 2:}
\begin{verbatim}
class Hammer(Tool):
    def __init__(self, name, weight):
        Tool.__init__()    # Error: missing 'name' argument
        self.weight = weight
\end{verbatim}

\textbf{Fix:}
\begin{verbatim}
class Hammer(Tool):
    def __init__(self, name, weight):
        super().__init__(name)
        self.weight = weight
\end{verbatim}

\newpage

\section*{Part 3 – Code Writing (30 points)}

\subsection*{1. Book and Textbook Classes}

\begin{verbatim}
class Book:
    def __init__(self, title, author):
        self.title = title
        self.author = author

class Textbook(Book):
    def __init__(self, title, author, subject):
        super().__init__(title, author)
        self.subject = subject

    def __str__(self):
        return f"{self.title} by {self.author} [Subject: {self.subject}]"
\end{verbatim}

\subsection*{2. Movie Class with Comparison Operators}

\begin{verbatim}
class Movie:
    def __init__(self, title, rating):
        self.title = title
        self.rating = rating

    def __eq__(self, other):
        return self.title == other.title

    def __lt__(self, other):
        return self.rating < other.rating
\end{verbatim}

\subsection*{3. Mixin and Inheritance}

\begin{verbatim}
class TimestampMixin:
    def timestamp(self):
        print("Logged at some time")

class Document:
    def __init__(self, filename):
        self.filename = filename

class PDFDocument(Document, TimestampMixin):
    def print_info(self):
        print(f"File: {self.filename}")
        self.timestamp()
\end{verbatim}

\newpage

\section*{Part 4 – Comment the Code (10 points)}

\begin{verbatim}
class Animal:
    def __init__(self, species="unknown"):       # (1) Constructor sets species; uses a default value
        self._species = species                  # Stored as a protected attribute

    def get_species(self):                        # (2) Getter method to access species attribute
        return self._species

    def speak(self):                              # (3) Base class placeholder for sound; can be overridden
        return "..."

class Dog(Animal):
    def speak(self):                              # (4) Overrides speak to return "Woof!" for Dog
        return "Woof!"
\end{verbatim}

\end{document}
