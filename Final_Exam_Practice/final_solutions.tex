\documentclass[12pt]{article}
\usepackage{amsmath, amssymb}
\usepackage{fancyhdr}
\usepackage{fullpage}
\usepackage{listings}
\usepackage{enumitem}
\usepackage{graphicx}
\usepackage{courier}
\usepackage{float}

\pagestyle{fancy}
\fancyhf{}
\rhead{CPSC 250 - Final Exam Review Solutions}
\lhead{Programming for Data Manipulation}
\rfoot{Page \thepage}

\title{CPSC 250 - Programming for Data Manipulation \\ Final Exam Review – Solutions}
\date{}
\begin{document}
\maketitle

\section*{Part I: Multiple Choice (30 points)}

\begin{enumerate}[label=\arabic*.]
\item \textbf{C} – A single-element tuple must use a trailing comma: \texttt{(5,)} is a tuple.
\item \textbf{C} – \texttt{\_\_lt\_\_} implements the less-than operator \texttt{<}.
\item \textbf{D} – \texttt{x is y} is True because they point to the same list object.
\item \textbf{A} – \texttt{plt.hist()} creates a histogram from data.
\item \textbf{B} – Encapsulation hides internal state and data using private attributes.
\item \textbf{C} – Default arguments must come after required ones.
\item \textbf{A} – \texttt{df.loc[0]} returns the first row (as a Series).
\item \textbf{B} – In a balanced BST, search is \texttt{O(log n)}.
\item \textbf{B} – \texttt{OLS()} from \texttt{statsmodels.api} performs regression.
\item \textbf{B} – The derived class method overrides the base method.
\item \textbf{B} – Inheritance lets a class reuse code from another class.
\item \textbf{A} – Python does not support true overloading, but default args allow flexibility.
\item \textbf{C} – The set comprehension eliminates duplicates: \texttt{\{1, 2, 3\}}.
\item \textbf{C} – Tuples are immutable.
\item \textbf{C} – \texttt{np.linspace()} generates evenly spaced values.

\end{enumerate}

\newpage
\section*{Part II: Error Identification (15 points)}

\textbf{Original Buggy Code:}
\begin{lstlisting}[language=Python]
def count_words(filename):
    with open(filename) as f:
        words = f.read().split()

    counts = {}
    for word in words:
        if word in counts:
            counts[word] = 1
        else:
            counts[word] += 1

    return counts
\end{lstlisting}

\textbf{Fixed Code:}
\begin{lstlisting}[language=Python]
def count_words(filename):
    with open(filename) as f:
        words = f.read().split()

    counts = {}
    for word in words:
        if word in counts:
            counts[word] += 1
        else:
            counts[word] = 1

    return counts
\end{lstlisting}

\textbf{Explanation:}  
The original logic reversed the update: it reset the count to 1 if the word already existed. Also, the default case incorrectly tried to increment a non-existent key.

\newpage
\section*{Part III: Code Writing (30 points)}

\textbf{Q1. Recursive factorial (6 points)}
\begin{lstlisting}[language=Python]
def factorial(n):
    if n <= 1:
        return 1
    else:
        return n * factorial(n - 1)
\end{lstlisting}

---

\textbf{Q2. Book class with \_\_str\_\_ (8 points)}
\begin{lstlisting}[language=Python]
class Book:
    def __init__(self, title, author):
        self.__title = title
        self.__author = author

    def __str__(self):
        return f'"{self.__title}" by {self.__author}'
\end{lstlisting}

---

\textbf{Q3. Read CSV and plot (8 points)}
\begin{lstlisting}[language=Python]
import pandas as pd
import matplotlib.pyplot as plt

def read_and_plot(filename):
    df = pd.read_csv(filename)
    plt.scatter(df['x'], df['y'])
    plt.xlabel("x")
    plt.ylabel("y")
    plt.title("Scatterplot of y vs x")
    plt.show()
\end{lstlisting}

---

\textbf{Q4. Shape polymorphism (8 points)}
\begin{lstlisting}[language=Python]
class Shape:
    def area(self):
        return 0

class Square(Shape):
    def __init__(self, side):
        self.side = side

    def area(self):
        return self.side ** 2

class Triangle(Shape):
    def __init__(self, base, height):
        self.base = base
        self.height = height

    def area(self):
        return 0.5 * self.base * self.height

# Demonstrate polymorphism
shapes = [Square(4), Triangle(3, 6), Square(2)]
for s in shapes:
    print(s.area())
\end{lstlisting}

\newpage
\section*{Part IV: Code Comprehension + Commenting (25 points)}

\textbf{Original Function:}
\begin{lstlisting}[language=Python]
import statsmodels.api as sm
import pandas as pd

def regress(df, yname, xnames):
    X = df[xnames]                  # Select independent variables
    X = sm.add_constant(X)         # Add intercept term
    y = df[yname]                  # Select dependent variable
    model = sm.OLS(y, X)           # Create OLS regression model
    results = model.fit()          # Fit the model
    print(results.summary())       # Print regression summary
\end{lstlisting}

\textbf{Explanation:}  
This function performs a multiple linear regression using the specified dependent variable (`yname`) and a list of independent variables (`xnames`) from a DataFrame. It adds a constant term (intercept), fits the model using OLS (ordinary least squares), and prints a detailed summary with coefficients, R-squared value, and p-values.

\end{document}
