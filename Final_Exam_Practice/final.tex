\documentclass[12pt]{article}
\usepackage{amsmath, amssymb}
\usepackage{fancyhdr}
\usepackage{fullpage}
\usepackage{listings}
\usepackage{enumitem}
\usepackage{graphicx}
\usepackage{courier}
\usepackage{float}
\usepackage{hyperref}

\pagestyle{fancy}
\fancyhf{}
\rhead{CPSC 250 - Final Exam Review}
\lhead{Programming for Data Manipulation}
\rfoot{Page \thepage}

\title{CPSC 250 - Programming for Data Manipulation\\Final Exam Review Version}
\date{}
\begin{document}
\maketitle

\section*{Instructions}
You have 150 minutes. This is a closed-book, in-class review exam. You may not use ChatGPT or any programming tools. Answer clearly and legibly. Show all work.

\section*{Part I: Multiple Choice (30 points, 2 points each)}
Circle the best answer.

\begin{enumerate}[label=\arabic*.]

\item What is the output of the expression \texttt{type((5,))} in Python?  
\begin{enumerate}[label=\Alph*.]
\item \texttt{int}
\item \texttt{list}
\item \texttt{tuple}
\item \texttt{set}
\end{enumerate}

\item What does the \texttt{\_\_lt\_\_} method define for a class?  
\begin{enumerate}[label=\Alph*.]
\item Equality
\item String conversion
\item Less-than comparison
\item Logical negation
\end{enumerate}

\item Given \texttt{x = [1, 2, 3]} and \texttt{y = x}, which statement is true?  
\begin{enumerate}[label=\Alph*.]
\item \texttt{x} and \texttt{y} refer to different lists
\item \texttt{y} is a shallow copy of \texttt{x}
\item \texttt{x == y} is False
\item \texttt{x is y} is True
\end{enumerate}

\item What does \texttt{plt.hist(data)} do?  
\begin{enumerate}[label=\Alph*.]
\item Plots a histogram of \texttt{data}
\item Computes the mean of \texttt{data}
\item Sorts \texttt{data}
\item Removes outliers from \texttt{data}
\end{enumerate}

\item What is encapsulation in OOP?  
\begin{enumerate}[label=\Alph*.]
\item Storing data in lists
\item Restricting access to internal object details
\item Using recursion inside a class
\item Replacing operators like \texttt{+} and \texttt{==}
\end{enumerate}

\item Which of the following is true about default arguments in Python functions?  
\begin{enumerate}[label=\Alph*.]
\item Mutable defaults like lists are always safe to use  
\item Default values are evaluated each time the function is called  
\item Default arguments must appear after non-default ones  
\item You cannot use keyword arguments with default values  
\end{enumerate}

\item What does \texttt{df.loc[0]} return for a DataFrame \texttt{df}?  
\begin{enumerate}[label=\Alph*.]
\item First row
\item First column
\item First value
\item Column names
\end{enumerate}

\item In binary search trees, what is the time complexity of search in a balanced tree?  
\begin{enumerate}[label=\Alph*.]
\item \texttt{O(1)}
\item \texttt{O(log n)}
\item \texttt{O(n)}
\item \texttt{O(n log n)}
\end{enumerate}

\item Which method from \texttt{statsmodels.api} is used to perform OLS regression?  
\begin{enumerate}[label=\Alph*.]
\item \texttt{lm()}
\item \texttt{OLS()}
\item \texttt{linreg()}
\item \texttt{curve\_fit()}
\end{enumerate}

\item What happens if a base class method is overridden in a derived class?  
\begin{enumerate}[label=\Alph*.]
\item The base class method runs first  
\item The derived method replaces the base version  
\item The method becomes private  
\item Python raises an error  
\end{enumerate}

\item Which of the following best describes inheritance in Python?  
\begin{enumerate}[label=\Alph*.]
\item Classes cannot inherit from multiple parents  
\item Inheritance allows one class to reuse code from another  
\item Derived classes must override all methods  
\item Only abstract classes can be inherited  
\end{enumerate}

\item Which of the following best describes method overloading in Python?  
\begin{enumerate}[label=\Alph*.]
\item Not supported directly, but can be simulated with default arguments  
\item Multiple methods with the same name are allowed  
\item Python only allows one constructor  
\item You must use decorators for it to work  
\end{enumerate}

\item What does the following expression return?  
\texttt{\{x for x in [1, 2, 2, 3, 3, 3]\}}  
\begin{enumerate}[label=\Alph*.]
\item \texttt{[1, 2, 3]}
\item \texttt{\{1, 2, 2, 3, 3, 3\}}
\item \texttt{\{1, 2, 3\}}
\item A list of tuples
\end{enumerate}

\item Which of the following is true about tuples in Python?  
\begin{enumerate}[label=\Alph*.]
\item They are mutable  
\item They are slower than lists  
\item They are immutable  
\item They cannot be nested  
\end{enumerate}

\item What is the primary use of \texttt{np.linspace()}?  
\begin{enumerate}[label=\Alph*.]
\item Count elements in a list  
\item Sort an array  
\item Create a sequence of evenly spaced values  
\item Generate random numbers  
\end{enumerate}

\end{enumerate}

\newpage
\section*{Part II: Error Identification (15 points)}

The following code is intended to create a dictionary mapping each unique word in a file to the number of times it appears. Find and correct all errors.

\begin{lstlisting}[language=Python]
def count_words(filename):
    with open(filename) as f:
        words = f.read().split()

    counts = {}
    for word in words:
        if word in counts:
            counts[word] = 1
        else:
            counts[word] += 1

    return counts
\end{lstlisting}

\vspace{3in} % space for writing

\newpage
\section*{Part III: Code Writing (30 points)}

\textbf{Q1. (6 points)}  
Write a function \texttt{factorial(n)} that uses recursion to compute \texttt{n!}.

\vspace{2in}

\textbf{Q2. (8 points)}  
Write a class \texttt{Book} with private fields for \texttt{title} and \texttt{author}, and a \texttt{\_\_str\_\_} method that returns \texttt{"<title>" by <author>}.

\vspace{2.5in}

\textbf{Q3. (8 points)}  
Write a function \texttt{read\_and\_plot(filename)} that reads a CSV file with columns \texttt{x} and \texttt{y} and produces a scatterplot using matplotlib.

\vspace{2.5in}

\textbf{Q4. (8 points)}  
Write a base class \texttt{Shape} with a method \texttt{area()} that returns 0. Then write derived classes \texttt{Square} and \texttt{Triangle}, each with their own constructor and overridden \texttt{area()} method. Demonstrate polymorphism using a list of \texttt{Shape} objects.

\vspace{3in}

\newpage
\section*{Part IV: Code Comprehension + Commenting (25 points)}

The following function is meant to perform simple multi-variable regression using \texttt{statsmodels}. Add comments explaining each line and describe what this code is doing.

\begin{lstlisting}[language=Python]
import statsmodels.api as sm
import pandas as pd

def regress(df, yname, xnames):
    X = df[xnames]
    X = sm.add_constant(X)
    y = df[yname]
    model = sm.OLS(y, X)
    results = model.fit()
    print(results.summary())
\end{lstlisting}

\vspace{3.5in}

\textbf{What does the function output represent?}

\vspace{1.5in}

\end{document}
