\documentclass[11pt]{article}
\usepackage{amsmath}
\usepackage{amssymb}
\usepackage{listings}
\usepackage{fancyhdr}
\usepackage{geometry}
\geometry{margin=1in}
\usepackage{enumitem}
\usepackage{multicol}

\pagestyle{fancy}
\fancyhf{}
\rhead{CPSC 250 – Test 2}
\lhead{Programming for Data Manipulation}
\cfoot{\thepage}

\title{\vspace{-1cm}CPSC 250 – Programming for Data Manipulation – Test 2}
\date{}
\begin{document}
\maketitle

\noindent\textbf{Time:} 75 minutes \\
\textbf{Instructions:} Answer all questions on this test paper. No electronic devices or notes are permitted. Be sure to write clearly.

\vspace{0.5cm}
\section*{Part A: Multiple Choice (10 points, 2 points each)}
\noindent Circle the one correct answer.

\begin{enumerate}[label=\arabic*.]
    \item What is the result of modifying an element of a list passed to a function? \\
    \begin{tabular}{ll}
        A. & It raises a TypeError. \\
        B. & Only a local copy is changed. \\
        C. & The original list is changed. \\
        D. & The function cannot access the list.
    \end{tabular}

    \item What is the purpose of the \verb|__init__| method in a Python class? \\
    \begin{tabular}{ll}
        A. & It defines a string representation of the object. \\
        B. & It is used to compare objects. \\
        C. & It initializes the object’s state. \\
        D. & It deletes the object.
    \end{tabular}

    \item What does Python’s garbage collector do? \\
    \begin{tabular}{ll}
        A. & Prevents memory leaks by closing files. \\
        B. & Automatically deletes unused variables after each loop. \\
        C. & Frees memory by deleting objects with zero references. \\
        D. & Ensures all objects are copied when passed to functions.
    \end{tabular}

    \item What is one advantage of using setter methods instead of direct attribute access? \\
    \begin{tabular}{ll}
        A. & Setter methods reduce the number of attributes in a class. \\
        B. & Setter methods automatically call \verb|__str__()| when an attribute changes. \\
        C. & Setter methods can include validation logic before updating a value. \\
        D. & Setter methods are faster than direct access.
    \end{tabular}

    \item Which of the following correctly overloads the equality (\verb|==|) operator in a class? \\
    \begin{tabular}{ll}
        A. & \verb|def __eq__(x, y):| \\
        B. & \verb|def __equal__(self, other):| \\
        C. & \verb|def __compare__(self, other):| \\
        D. & \verb|def __eq__(self, other):|
    \end{tabular}
\end{enumerate}

\newpage

\section*{Part B: Find the Error (15 points, 3 points each)}
Each of the following code snippets has one or more errors. Identify and explain them.

\begin{enumerate}[label=\arabic*.]
    \item \begin{lstlisting}[language=Python]
def double(n):
    n *= 2

value = 7
result = double(value)
print(result)
    \end{lstlisting}

    \item \begin{lstlisting}[language=Python]
class Student:
    def __init__(self, name, grade):
        name = name
        grade = grade
    \end{lstlisting}

    \item \begin{lstlisting}[language=Python]
def add_name(names=None):
    names.append("Alice")
    return names
    \end{lstlisting}

    \item \begin{lstlisting}[language=Python]
class Counter:
    def __init__(self):
        self.count = 0

    def __str__():
        return f"Count is {self.count}"
    \end{lstlisting}

    \item \begin{lstlisting}[language=Python]
class Vector:
    def __init__(self, x, y):
        self.x = x
        self.y = y

    def __add__(self, v):
        return (self.x + v.x, self.y + v.y)
    \end{lstlisting}
\end{enumerate}

\newpage

\section*{Part C: Code Writing (30 points)}

\begin{enumerate}[label=\arabic*.]
    \item (10 points) Write a function \verb|modify_list(mylist)| that appends the value \verb|42| to the list. Call the function on a list from main code, and explain why the original list is or is not changed.

    \vspace{20cm}

    \item (10 points) Write a class called \verb|Book| that has:
    \begin{itemize}
        \item Three instance variables: \verb|title|, \verb|author|, and \verb|pages|
        \item A constructor
        \item Getters and setters for all variables
        \item A method \verb|summary()| that returns a description string
        \item A \verb|__str__()| method that returns something like \verb|"Book: Title by Author (300 pages)"|
    \end{itemize}

    \vspace{20cm}

    \item (10 points) Modify your \verb|Book| class to support the \verb|+| operator to combine two books into a new book:
    \begin{itemize}
        \item Title = "Collection"
        \item Author = "Various"
        \item Pages = sum of both
    \end{itemize}

    \vspace{8cm}
\end{enumerate}

\newpage

\section*{Part D: Code Commentary (20 points)}  
The following program creates a class representing an inventory item. \textbf{Write comments next to each line} explaining what it does.

\begin{lstlisting}[language=Python]
class InventoryItem:

    def __init__(self, name, quantity):
        self.name = name
        self.quantity = quantity


    def restock(self, amount):
        self.quantity += amount


    def sell(self, amount):
        if amount > self.quantity:
            print("Not enough in stock")
        else:
            self.quantity -= amount


    def __str__(self):
        return f"{self.name}: {self.quantity} in stock"
\end{lstlisting}

\end{document}
