\documentclass[12pt]{article}
\usepackage{amsmath}
\usepackage{enumitem}
\usepackage{fancyhdr}
\usepackage[margin=1in]{geometry}
\usepackage{courier}
\usepackage{listings}
\usepackage{multicol}
\usepackage{xcolor}

\pagestyle{fancy}
\fancyhf{}
\rhead{CPSC 250 – Python for Data Manipulation – Summer 2025 Test 1 – Solutions}
\lhead{Name:\underline{\hspace{6cm}}}
\rfoot{Page \thepage}

\lstset{
  basicstyle=\ttfamily\small,
  breaklines=true,
  keywordstyle=\color{blue},
  commentstyle=\color{gray},
  stringstyle=\color{orange},
}

\begin{document}

\begin{center}
    \Large \textbf{CPSC 250 – Python for Data Manipulation – Summer 2025 Test 1} \\
    \large \textbf{Solution Set} \\
    \normalsize Topics: Primitive Data Types, Lists \& Tuples, Branching, Loops
\end{center}

\vspace{0.5cm}

\section*{Part I – Multiple Choice (10 points)}

\begin{enumerate}[label=\arabic*.]
    \item \textbf{B} \quad Integer division truncates the decimal, so \texttt{10 // 3} is 3.

    \item \textbf{A} \quad Lists use square brackets; tuples use parentheses, sets use curly braces, strings use quotes.

    \item \textbf{A} \quad \texttt{int("5")} converts string "5" to integer 5, so 5 + 3 = 8.

    \item \textbf{C} \quad Both \texttt{x >= 10 and x <= 20} and \texttt{10 <= x <= 20} correctly check the range.

    \item \textbf{C} \quad Tuples are zero-indexed; \texttt{t[2]} gives the 3rd element, which is 3.
\end{enumerate}

\section*{Part II – Find the Error (15 points)}

\begin{enumerate}[resume,label=\arabic*.]
    \item Error: Index out of range (list has indices 0 to 3). Correction:
\begin{lstlisting}
numbers = [10, 20, 30, 40]
print(numbers[3])  # Change 4 to 3
\end{lstlisting}

    \item Error: Missing colon after \texttt{for} statement. Correction:
\begin{lstlisting}
for i in range(1, 5):
    print(i * 2)
\end{lstlisting}

    \item Error: Missing indentation inside \texttt{while} loop. Correction:
\begin{lstlisting}
count = 5
while count > 0:
    print(count)
    count = count - 1
\end{lstlisting}
\end{enumerate}

\section*{Part III – Code Writing (20 points)}

\begin{enumerate}[resume,label=\arabic*.]
    \item Sample solution:

\begin{lstlisting}
total = 0
for _ in range(10):
    num = int(input("Enter an integer: "))
    if num % 3 == 0:
        total += num
print("Sum of numbers divisible by 3:", total)
\end{lstlisting}

    \item Sample solution:

\begin{lstlisting}
for i in range(1, 11):
    print(i**2)
\end{lstlisting}
\end{enumerate}

\section*{Part IV – Code Comprehension and Commenting (15 points)}

\begin{lstlisting}
values = [2, 5, 9, 12, 3]      # List of numbers
product = 1                    # Initialize product accumulator
for v in values:               # Loop through each number in list
    if v > 4:                  # Check if current number is greater than 4
        product *= v           # Multiply product by current number if condition is true
print("Product of values greater than 4 is", product)  # Output the final product
\end{lstlisting}

\section*{Part V – Short Answer (10 points)}

\begin{enumerate}[resume,label=\arabic*.]
    \item Accessing index 5 of a list with only 3 elements causes an \texttt{IndexError} because valid indices are 0, 1, 2. For example:
\begin{lstlisting}
lst = [10, 20, 30]
print(lst[5])  # IndexError: list index out of range
\end{lstlisting}

    \item Tuples should be used instead of lists when the data should be immutable (not changeable). For example, storing fixed configuration values:
\begin{lstlisting}
coordinates = (10.0, 20.5)  # Tuple representing fixed (x, y) coordinates
\end{lstlisting}
\end{enumerate}

\section*{Total: \underline{\hspace{2cm}} / 70}

\end{document}

