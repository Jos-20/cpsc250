\documentclass[12pt]{article}
\usepackage{amsmath}
\usepackage{enumitem}
\usepackage{fancyhdr}
\usepackage[margin=1in]{geometry}
\usepackage{courier}
\usepackage{listings}
\usepackage{multicol}
\usepackage{xcolor}

\pagestyle{fancy}
\fancyhf{}
\rhead{CPSC 250 – Python for Data Manipulation – Summer 2025 Test 1}
\lhead{Name:\underline{\hspace{6cm}}}
\rfoot{Page \thepage}

\lstset{
  basicstyle=\ttfamily\small,
  breaklines=true,
  keywordstyle=\color{blue},
  commentstyle=\color{gray},
  stringstyle=\color{orange},
}

\begin{document}

\begin{center}
    \Large \textbf{CPSC 250 – Python for Data Manipulation – Summer 2025 Test 1} \\
    \normalsize Topics: Primitive Data Types, Lists \& Tuples, Branching, Loops
\end{center}

\vspace{0.5cm}

\section*{Part I – Multiple Choice (10 points)}

Circle the best answer. Each question is worth 2 points.

\begin{enumerate}[label=\arabic*.]
    \item What is the result of \texttt{10 // 3}?
    \begin{enumerate}[label=\Alph*.]
        \item 3.33
        \item 3
        \item 4
        \item 1
    \end{enumerate}

    \item Which of the following is a valid Python list?
    \begin{enumerate}[label=\Alph*.]
        \item \texttt{\char`\[1, 2, 3\char`\]}
    \item \texttt{(1, 2, 3)}
    \item \texttt{\{1, 2, 3\}}
    \item \texttt{"1, 2, 3"}

    \end{enumerate}

    \item What is the result of \texttt{int("5") + 3}?
    \begin{enumerate}[label=\Alph*.]
        \item 8
        \item "53"
        \item Error
        \item 5
    \end{enumerate}

    \item Which of the following correctly checks if \texttt{x} is between 10 and 20 (inclusive)?
    \begin{enumerate}[label=\Alph*.]
        \item \texttt{x >= 10 and x <= 20}
        \item \texttt{10 <= x <= 20}
        \item Both A and B
        \item Neither A nor B
    \end{enumerate}

    \item What is the output of the following code? \texttt{t = (1, 2, 3, 4); print(t[2])}
    \begin{enumerate}[label=\Alph*.]
        \item 1
        \item 2
        \item 3
        \item 4
    \end{enumerate}
\end{enumerate}

\section*{Part II – Find the Error (15 points)}

Each of the following code snippets contains one or more errors. Identify and correct them.

\begin{enumerate}[resume,label=\arabic*.]
    \item
\begin{lstlisting}
numbers = [10, 20, 30, 40]
print(numbers[4])
\end{lstlisting}

    \item
\begin{lstlisting}
for i in range(1, 5)
print(i * 2)
\end{lstlisting}

    \item
\begin{lstlisting}
count = 5
while count > 0:
print(count)
count = count - 1
\end{lstlisting}
\end{enumerate}

\section*{Part III – Code Writing (20 points)}

Write Python code for each of the following tasks.


\begin{enumerate}[resume,label=\arabic*.]
    \item Write Python code that asks the user to enter 10 integers, then calculates and prints the sum of all numbers divisible by 3.

    \vspace{12cm}

    \item Write Python code that uses a loop to print the squares of the numbers from 1 to 10 inclusive, each on its own line.

    \vspace{9cm}
\end{enumerate}



\section*{Part IV – Code Comprehension and Commenting (15 points)}

The following program is complete but has no comments. Write comments to explain what each line does.

\begin{lstlisting}
values = [2, 5, 9, 12, 3]
product = 1
for v in values:
    if v > 4:
        product *= v
print("Product of values greater than 4 is", product)
\end{lstlisting}

\vspace{5cm}

\section*{Part V – Short Answer (10 points)}

\begin{enumerate}[resume,label=\arabic*.]
    \item What happens if you try to access index 5 of a list with only 3 elements? Give an example and explain.

    \vspace{3cm}

    \item When should you use a tuple instead of a list in Python? Give an example.

    \vspace{3cm}
\end{enumerate}

\section*{Total: \underline{\hspace{2cm}} / 70}

\end{document}

